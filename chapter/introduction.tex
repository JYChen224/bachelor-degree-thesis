%# -*- coding: utf-8-unix -*-
\chapter{简介}\label{chap:intro}

这是南京航空航天大学(非官方)本科生学位论文\LaTeX 模板,当前版本是\version。本模板由\nuaathesis~Group共同开发,模板文档由Old Jack撰写。

本模板最早可以追溯到人人网上的一篇博客\footnote{\url{http://bit.ly/2rGOhf6}},由黄大宁、邓欣珂、徐添豪、石坤四人共同开发完善,参考了当时东南大学的\seuthesix 模板;除此之外在Github上也可以找到一个repo\footnote{\url{https://github.com/nuaa803/nuaa-thesis}},由Felix Ding、Jun Wang、Jackie Hou三位老师和Vevi Zhong同学共同维护,但是repo中的.cls和.sty文件是空文件。

回顾人人网的模板,没有直接提供nuaa.png和nuaa.bst文件,可以使用强制编译的方法生成文件,但是缺少左上角南航字样,参考文献格式也不符合标准。除此之外,旧模板使用了已经被放弃使用的CJK宏包,因此在编译\verb+\Unicode{}+命令时会出错,代码的阅读性和维护性也不如现在的ctex和xeCJK。由于上述原因,许多初次使用\LaTeX 和使用经验不多的同学,在一开始就放弃了使用旧版模板进行毕业设计的书写及排版。

基于南航无可用\LaTeX 学位模板可用的现状,\nuaathesis~Group基于旧\oldnuaathesis 模板、现东南大学的\seuthesix 模板和上海交通大学的SJTU Thesis模板,对模板进行了二次开发,基本实现了学士学位论文的模板。\nuaathesis~Group现有成员短期内不计划开发团队报告、硕士博士学位论文等其他文档的\LaTeX 模板,留给后续成员以及其他有需要、有能力的南航学子以后开发。

现在\nuaathesis 模板的代码托管在Github\footnote{\url{https://github.com/jackwzh/nuaathesis}}上,如有修改建议或者其他要求欢迎在Github上开issue或提pull request,\nuaathesis~Group会尽快回复,并酌情处理您的要求。

本模板基于Windows~10平台开发,使用MiKTeX v2.9发行版,所使用的宏包均跟进到最新版本。Linux平台由张一白使用\TeX Live测试,macOS平台由王成欣进行了测试,目前尚未出现任何问题。本模板尚未在Windows平台使用\CTeX / \TeX Live进行测试,如出现问题,请自行Google、Bing、Baidu搜索解决方法。学会使用搜索引擎、熟练阅读外文是一个学生最基本的能力,更是一个\LaTeX 使用者得以立足和前进的根本。

\nuaathesis~Group非常欢迎有其他南航的\LaTeX 使用者加入到本模板的开发与维护当中来,不断完善模板,为南航广大学子造福!

\section{模板使用}
\subsection{准备工作}
\begin{itemize}
  \item \TeX 发行版:Windows 系统推荐使用MiKTeX和\TeX Live这两种发行版,前者占用空间小,只在有宏包缺失情况下才进行下载,后者占用空间大,但基本无需担心宏包缺失。Linux系统(Arch系除外)推荐手动安装\TeX Live发行版,官方源中的TeXLive版本跟进较慢。macOS可以参考SJTU Thesis中的介绍。
  \item \TeX 知识:本说明文档提供\TeX 使用的例子,但不能解决所有的问题,因此使用前请自行学习\TeX~\&~\LaTeX 相关知识。
\end{itemize}

\subsection{模板编译}

\textbf{切记使用\XeLaTeX 进行编译。}使用\XeLaTeX 和\hologo{BibTeX}在文档中加入参考文献的流程可参考如下命令:

\begin{lstlisting}[basicstyle=\small\ttfamily, caption=手动逐次编译, numbers=none]
xelatex -no-pdf .tex文件名
biber --debug .tex文件名
xelatex .tex文件名
xelatex .tex文件名
\end{lstlisting}

使用\XeLaTeX 引擎编译可以直接通过各\LaTeX 编辑器实现,如:TeXworks,TeXmaker,TeXStudio,Emacs+插件,Atom+插件等等。biber命令需使用Windows的cmd/Power Shell、Linux和macOS下的bash实现。Windows平台可以自行编写简单的.bat批处理文件来实现。

使用biber需注意:\textbf{.bib文件内的文件记录必须在.tex文件中被引用,不引用的记录不要因为懒而不去除,否则将编译失败}

目录内容需要编译两次才能正常显示,原因推断为早期的电脑内存不够,所以将目录的生成分成了两步来进行。

\subsection{模板文件结构}
\begin{itemize}[noitemsep,topsep=0pt,parsep=0pt,partopsep=0pt]
  \item .tex文件:主文件,chapter下有各个章节的文件,强烈建议将文章模块化,方便调试与版本管理。
  \item .cls、.cfg文件:模板定义文件
  \item .bib文件:参考文献数据库文件
  \item figure文件夹:存放要插入的图片,其中nuaa.png不可删除
\end{itemize}
