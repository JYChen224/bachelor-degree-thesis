\chapter{结论}
\section{研究内容总结}

本文主要对踝关节外骨骼的控制系统进行了研究,主要工作总结如下:

(1)搭建了外骨骼的传感系统。针对所研究的踝关节式外骨骼,搭建了基于应变片的外骨骼力矩测量模块、基于足底开关的步态周期检测模块、基于EMG信号和肌肉激活度检测模块、基于IMU的人体姿态数据采集系统。

(2)设计了外骨骼力矩控制系统。力矩控制由上层控制器与底层控制器组成,上层控制器用来产生期望力矩曲线,底层控制器实现对期望力矩曲线的跟踪。针对上层控制器,本文采用基于时间的直接力矩控制器,通过三次函数插值得到期望力矩曲线。针对底层控制器,本文提出了三种控制算法,设计了四种不同组合形式的控制器。

(3)研究了“人在环中”的外骨骼优化方法。针对外骨骼助力模式因人而异的问题,采集穿戴者的生理信号做目标函数,对助力参数进行优化,通过迭代寻找最适合穿戴者的助力模式,从而充分发挥外骨骼潜能。本文采用贝叶斯优化,并选取肌肉激活指标作生理反馈,通过实验验证了优化的有效性。

\section{课题展望}

本文对外骨骼系统中的传感、控制与优化问题进行了一系列基础工作,在此基础上对未来的研究工作有如下几点展望:

(1)搭建多传感器融合的外骨骼系统。控制系统非常依赖由传感系统得到的反馈信息,因此可靠、准确、稳定的传感数据对外骨骼系统而言至关重要。本文研究过程中多次出现由传感器数据不稳定而导致的系统故障,后续会从多传感器数据融合的角度出发,搭建更为可靠的传感系统。

(2)尝试多种上层力矩控制策略。本文研究的期望力矩曲线是基于时间的函数,只能作为稳态行走时的助力模式。后面将进一步研究在不同上层控制器下力矩控制的效果。

(3)“人在环中”优化最为一个较新的研究领域,至今还有很多问题函待解决。本文的优化研究最为一个尝试,初步证明了基于肌电的生理反馈可以用于优化。但更合适的生理反馈形式,更高效的优化算法,仍有待进一步探讨。