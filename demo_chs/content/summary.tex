\chapter{结论}
\section{研究内容总结}

本文主要对踝关节外骨骼的控制系统进行了研究,主要工作总结如下:

(1)搭建了外骨骼的传感系统。

针对所研究的踝关节式外骨骼搭建了一套传感系统,通过应变片测量外骨骼的人机交互力矩,使用足底开关检测人体的步态周期,设计了基于IMU的人体姿态测量系统,实现了基于sEMG信号的肌肉激活度测量。

(2)设计了外骨骼力矩控制系统。

力矩控制由上层控制器与底层控制器组成,上层控制器用来产生期望力矩曲线,底层控制器实现对期望力矩曲线的跟踪。针对上层控制器,本文采用基于时间的直接力矩控制器,通过三次函数插值得到期望力矩曲线。针对底层控制器,本文提出了三种控制算法,设计了四种不同组合形式的控制器。

(3)研究了“人在环中”的外骨骼优化方法。

针对外骨骼最佳助力模式因人而异的问题,研究了“人在环中”的外骨骼参数优化方法。本文以穿戴者行走时肌肉激活度为目标函数,使用贝叶斯优化外搜寻最佳助力参数。实验表明,外骨骼系统能够在五分钟内寻找到最佳助力参数。在最佳助力参数下,受试者肌肉激活度平均下降12.2\%。

\section{课题展望}

本文对外骨骼系统中的传感、控制与优化问题进行了一系列基础工作,在此基础上对未来的研究工作有如下几点展望:

(1)控制系统十分依赖由传感系统得到的反馈信息,可靠、准确、稳定的传感数据对外骨骼系统而言至关重要。本文研究过程中多次出现由传感器数据不稳定而导致的系统故障。后续会从多传感器数据融合的角度出发,搭建更为可靠的传感系统。

(2)本文研究的期望力矩曲线是基于时间的函数,只能作为稳态行走时的助力模式。后面将进一步研究具有动态特性的期望力矩曲线生成算法,并在不同上层控制器下验证力矩追踪控制的效果。

(3)“人在环中”优化最为一个较新的研究领域,至今还有很多问题函待解决。本文的优化研究最为一个尝试,初步证明了基于肌电信号的生理反馈可以用于人在环中优化。但更合适的生理反馈形式,更高效的优化算法,仍有待进一步探讨。