\nuaaset{
  title = {\nuaathesis{} 日文论文示例},
  author = {佚名},
  college = {\TeX{} 学院},
  advisers = {Knuth\quad 教授},
  % applydate = {二〇一六年三月}  % default to current date
  %
  % bachelor only
  major = {\LaTeX{} 科学与技术},
  studentid = {131810299},
  classid = {1318001},
  % master/doctor only
  majorsubject = {编程与艺术},
  researchfield = {轮子制造},
  libraryclassid = {H319},        % 中图分类号
  subjectclassid = {050211},      % 学科分类号
  thesisid = {1028712 16-S022},   % 论文编号
}

\nuaasetEn{
  author = {nuaatug},
  college = {College of \TeX},
  majorsubject = {Programming and Typesetting},
  advisers = {Professor~Knuth},
  degreefull = {Master of Arts},
  % applydate = {March, 8012}
}

\nuaasetJa{
  title = {テーゼ見本でも長い\linebreak タイトルが必要です}
}

\begin{abstract}
本文主要演示日文论文写作时的注意事项。

大部分中文 \LaTeX{} 的内容同样适用于日文,在此不再赘述。

\jpn{もし、ここに日本語で書きたいなら、\textbackslash jpn を使ってください。}
\end{abstract}
\keywords{日语, 注意事项}

\begin{abstractJa}
ここに論文の要旨をお書きください。

もちろん改行も可能です。
\end{abstractJa}
\keywordsJa{学位論文, \LaTeX}


% load packages, define global settings/macros

\theoremstyle{nuaaplain}
\nuaatheoremchapu{definition}{定義}
\nuaatheoremchapu{assumption}{仮設}

\usepackage{subfig}
\usepackage{rotating}
\usepackage[usenames,dvipsnames]{xcolor}
\usepackage{metalogo}
\usepackage{tikz}
\usepackage{pgfplots}
\usepackage{ifthen}
\usepackage{longtable}
\usepackage{siunitx}
\usepackage{listings}
\usepackage{multirow}
\usepackage{lipsum}

\lstdefinestyle{lstStyleBase}{%
  basicstyle=\small\ttfamily,
  aboveskip=\medskipamount,
  belowskip=\medskipamount,
  lineskip=0pt,
  boxpos=c,
  showlines=false,
  extendedchars=true,
  upquote=true,
  tabsize=2,
  showtabs=false,
  showspaces=false,
  showstringspaces=false,
  numbers=left,
  numberstyle=\footnotesize,
  linewidth=\linewidth,
  xleftmargin=\parindent,
  xrightmargin=0pt,
  resetmargins=false,
  breaklines=true,
  breakatwhitespace=false,
  breakindent=0pt,
  breakautoindent=true,
  columns=flexible,
  keepspaces=true,
  framesep=3pt,
  rulesep=2pt,
  framerule=1pt,
  backgroundcolor=\color{gray!5},
  stringstyle=\color{green!40!black!100},
  keywordstyle=\bfseries\color{blue!50!black},
  commentstyle=\slshape\color{black!60}}

\lstdefinestyle{lstStyleShell}{%
    style=lstStyleBase,
    frame=l,
    rulecolor=\color{blue},
    language=bash}

\lstdefinestyle{lstStyleLaTeX}{%
    style=lstStyleBase,
    frame=l,
    rulecolor=\color{cyan},
    language=[LaTeX]TeX}

\lstnewenvironment{latex}{\lstset{style=lstStyleLaTeX}}{}
\lstnewenvironment{shell}{\lstset{style=lstStyleShell}}{}


%\usetikzlibrary{external}
%\tikzexternalize % activate!

